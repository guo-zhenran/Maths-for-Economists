\section{Operations of Sets}
\subsection{Complement, Intersection and Union}
\begin{definition}[Complement, Intersection and Union]
    Let $A,B$ be sets, we define the following operations:
\begin{description}
    \item[Complement] The (relative) complement of set $B$ in set $A$ is defined as $A \setminus B = \{x : (x \in A) \land (x \notin B)\}$.
    \item[Intersection] The intersection of sets $A$ and $B$ is defined as $A \cap B = \{x : (x \in A) \land (x \in B)\}$.
    \item[Union] The union of sets $A$ and $B$ is defined as $A \cup B = \{x : (x \in A) \lor (x \in B)\}$.
\end{description}

By definition, one can easily check:
\begin{proposition}
    Let $A,B,C$ be sets, then 
    \begin{description}
        \item[Commutativity] $A \cap B = B \cap A$ and $A \cup B = B \cup A$.
        \item[Associativity] $(A \cap B) \cap C = A \cap (B \cap C)$ and $(A \cup B) \cup C = A \cup (B \cup C)$.
        \item[Distributivity] $A \cap (B \cup C) = (A \cap B) \cup (A \cap C)$ and $A \cup (B \cap C) = (A \cup B) \cap (A \cup C)$
        \item[de Morgan's Laws] $(A\cap B)^c=A^c \cup B^c$ and $(A\cup B)^c=A^c \cap B^c$
        \item[Subset] $A \subset B \iff A \cap B = A \iff A \cup B = B$.
    \end{description}
\end{proposition}
\end{definition}

\subsection{Power Set}
\begin{definition}[Power Set]
    Let $A$ be a set, define:
    $$2^A = \{X : X \subset A\}$$
\end{definition}
\subsection{Cartesian Product}
We define an ordered pair or a n-tuple $x = (x_1,x_2,\cdots,x_n)$ and the equity means that all components are equal. Denote $x_j := pr_j(x)$ and we call it the jth projection of $x$.

The Cartesian product is to describe the set of ordered pairs.
\begin{definition}[Cartesian Product]
    Let $A,B$ be sets, define:
    $$A \times B = \{(a,b) : a \in A, b \in B\}$$
    \,and by deduction,
    $$\prod_{i=1}^n A_i = (A_1\times \cdots \times A_{n-1})\times A_n$$
    \,Specially, when $A_i = A$ for all $i$, we denote:
    $$A^n = \prod_{i=1}^n A$$
\end{definition}

To practice logic, we show how to prove the following proposition.
\begin{proposition}
    Let $A,B$ be sets, then $A \times B = \emptyset \iff (A = \emptyset) \lor (B = \emptyset)$  
\end{proposition}
\begin{proof}
    ($\Rightarrow $) We prove by contradiction. Suppose $A\times B$ is non-empty, which implies there exists $(a,b) \in A\times B$, then by definition of Cartesian product, we have $a \in A$ and $b \in B$, contradicting the assumption.

    ($\Leftarrow $) This side can be similarly proved by contradiction.
\end{proof}

\subsection{Family of Sets}
We now extend the operation of intersection and union a little bit. Consider we're intersecting a series of sets $A,B,C,\cdots$, we first
rename the sets as $\{A_i\}_{i\in I}$, where $I$ is called an index set, and intersect them according to the index $i$, that is $$\bigcap_{i \in I} A_i = \{x : \forall i \in I, x \in A_i\}.$$
\,Similarly, for union, we have
$$\bigcup_{i \in I} A_i = \{x : \exists i \in I, x \in A_i\}.$$

\begin{proposition}
    Let $\{A_{\alpha}:\alpha \in I_{A}\}$ and $\{B_{\beta}:\beta \in I_{B}\}$ be families of subsets of a set $X$, then
\begin{enumerate}
    \item Associativity: $$\left(\bigcap_{\alpha \in I_A} A_{\alpha}\right) \cap \left(\bigcap_{\beta \in I_B} B_{\beta}\right) = \bigcap_{(\alpha,\beta) \in I_A \times I_B} (A_{\alpha} \cap B_{\beta})$$
    and
    $$\left(\bigcup_{\alpha \in I_A} A_{\alpha}\right) \cup \left(\bigcup_{\beta \in I_B} B_{\beta}\right) = \bigcup_{(\alpha,\beta) \in I_A \times I_B} (A_{\alpha} \cup B_{\beta})$$
    \item Distributivity: $$A \cap \left(\bigcup_{\alpha \in I_A} A_{\alpha}\right) = \bigcup_{\alpha \in I_A} (A \cap A_{\alpha})$$
    and 
    $$A \cup \left(\bigcap_{\alpha \in I_A} A_{\alpha}\right) = \bigcap_{\alpha \in I_A} (A \cup A_{\alpha})$$
    \item de Morgan's Laws: $$\left(\bigcap_{\alpha \in I_A} A_{\alpha}\right)^c = \bigcup_{\alpha \in I_A} A_{\alpha}^c$$
    and $$\left(\bigcup_{\alpha \in I_A} A_{\alpha}\right)^c = \bigcap_{\alpha \in I_A} A_{\alpha}^c$$
\end{enumerate}
\end{proposition}
\begin{proof}
    We show how to prove (3). A common trick we use when proving the equality of 2 sets is to show that each set is the subset of the other set.

    For $\subset$, $\forall a \in \left(\bigcap_{\alpha \in I_A} A_{\alpha}\right)^c$, since $a \in \bigcap_{\alpha \in I_A} A_{\alpha} \iff \forall \alpha \in I_A, a \in A_{\alpha}$,
    \,we have $a \in \left(\bigcap_{\alpha \in I_A} A_{\alpha}\right)^c \iff \exists \alpha_0 \in I_A, a \in A_{\alpha_0}^c$ by negation. Thus $a \in A_{\alpha_0}^c \subset \bigcup_{\alpha \in I_A} A_{\alpha}^c.$

    For the other side, there exists an $\alpha_0 \in I_A$ such that $a\in A_{\alpha_0}^c$. Notice that $\bigcap_{\alpha \in I_A} A_{\alpha} \subset A_{\alpha_0}$, we have $A_{\alpha_0}^c \subset \left(\bigcap_{\alpha \in I_A} A_{\alpha}\right)^c.$
\end{proof}