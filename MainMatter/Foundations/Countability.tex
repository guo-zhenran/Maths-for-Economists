\section{Countability}
For finite sets, we can easily count the element and say it's countable. But what if the set have infinite many elements?
Mathematicians believe that natural numbers can be counted and is the smallest infinite set. In this section we formally treat infinity, before which we introduce the axiom of choice.
\subsection{The Axiom of Choice}
The axiom of choice states that for a family of non-empty sets $\{A_i\}_{i\in I}$, the Cartesian product $\prod_{i\in I} A_i$ is non-empty.

That is to say there exists a choice function $f$ such that $\forall i \in I, f(i) \in A_i.$ 

\subsection{Cardinality}
The cardinality of a finite set is defined as the number of elements, that is $|A|=\# A$. For infinite sets, the definition is complicated and we turn to compare the cardinality
of 2 sets.
\begin{definition}[Equality of Cardinality]
    Two sets $A$ and $B$ are said to have the same cardinality if there exists a bijection from $A$ to $B$, denoted as $|A|=|B|$. 
    If there exists a subsection of $B$ and a bijection from $A$ to that subsection, we say $|A|\leq |B|$.
\end{definition}

\begin{proposition}
    Let $X, Y$ be sets, the following statements are equivalent:
    \begin{enumerate}
        \item $|X|\leq|Y|$
        \item there exists an injection from $X$ to $Y$.
        \item there exists a surjection from $Y$ to $X$.
    \end{enumerate}
\end{proposition}
\begin{proof}
    $(1)\implies (2)$ is done by definition. 
    
    To show $(2)\implies (3)$, let $g$ be a mapping from $f(X)$ to $X$ such that $g\circ f|_{f(X)} = \operatorname{Id}_{f(X)}$, which is a bijection. For a fixed $x \in X$, define
$$h(y)= \begin{cases}
    g(y), & y \in f(X)\\
    x, & y \notin f(X)
\end{cases}$$
\,which is a surjection from $Y$ to $X$.

$(3)\implies (1)$ needs the axiom of choice. Let $A_x := \{y \in Y : f(y) = x \}$ and $A_x$ is non-empty as $f$ is surjective. 
By the axiom of choice, there exists a choice function $h$ such that $h(x) \in A_x$. Notice that
$$x\neq x^\prime \implies (A_x \cap A_{x^{\prime}})=\emptyset \implies h(x) \neq h(x^\prime)$$
\,Then $h$ is an injection from $X$ to $Y$.
\end{proof}

Actually, $(|\cdot|, \leq)$ is a total ordered set. See the following theorems.
\begin{theorem}[Anti-symmetry and Completeness of Cardinality]
Let $X,Y$ be sets.
\begin{description}
    \item[Bernstein-Schroeder] $(|Y|\leq |X| \land |X|\leq |Y|) \implies |X|=|Y|$
    \item[Completeness] $|X| \leq |Y| \lor |Y| \leq |X|$
\end{description}    
\end{theorem}
\begin{note}
    Note that the completeness is equivalent to the axiom of choice.
\end{note}

\subsection{Countable Set}
\begin{definition}[Countability]
    A set $X$ is countable if $|X|=|\mathbb{N}|$.
\end{definition}

