With the concept of relation, we've added some structures to a set. For example, we've encountered $(\operatorname{Funct}(X),\circ)$ and we know that it's associative, it has an identity element $\operatorname{Id}_X$ and for bijective ones, each element has an inverse.

In this chapter, we'll go deeper into understanding the abstract properties of a set and the added relation. We introduce some basic algebraic concepts and important propositions.

\section{Group}
\subsection{the Definition}
\begin{definition}[Group]
    If, for the 2-tuple $(G,\odot)$, the relation $\odot$ is associative, then we call a semi-group. Additionally, if there exists an identity element, we call it a monoid. 
    
    A group is a monoid such that for each element $g\in G$, there exists an inverse element $g^{-1}\in G$ and we call it an Abelian group if the relation $\odot$ is commutative.
\end{definition}
\begin{note}
    Note that the identity element is unique by \propref{prop: uniqueness_identity}. The following 2 propositions are also useful:
    \begin{itemize}
        \item $\forall g \in G, g \odot h = g \iff h \odot g = g$ 
        \item $(g\odot h)^{-1} = h^{-1} \odot g^{-1}$
    \end{itemize}
\end{note}

\begin{definition}[Subgroup]
    For a nonempty subset $H$ of $G$, if it's closed under the operation $\odot$ \footnote{Meaning $$H\odot H \subset H : \iff \forall h_1,h_2 \in H : (h_1 \odot h_2) \in H.$$} and $\forall h \in H, h^{-1}\in H$, we say $(H,\odot)$ is a subgroup of $G$\footnote{We can simply denote it as $H$ if the operation is clear from the context.}.
We call $g\odot H$ and $H \odot g$ the left and right coset of $g \in G$ with respect to $H$ and if $g\odot H = H \odot g$, we say $H$ is a normal subgroup.
\end{definition}

\subsection{Homomorphisms}
Now we turn to mappping one group to the other. A very special kind is the one that preserves the algebraic structure and we call it a homomorphism.
\begin{definition}[Homomorphism]
    A homomorphism $\varphi$ is a mapping that maps $(G,\odot)$ to $(H,\circledast)$ such that $\forall g_1,g_2 \in G: \varphi(g_1 \odot g_2)= \varphi(g_1)\circledast\varphi(g_2)$. A homomorphism that maps $G$ to itslef is called an endomorphism.
\end{definition}

We list some important definitions and propositions.
\begin{proposition} \label{prop: identity2identity}
    A homomorphism maps the identity element to the identity element.
\end{proposition}
\begin{proof}
    Since $e_H \circledast \varphi(e_G) = \varphi(e_G \odot e_G) = \varphi(e_G)\circledast \varphi(e_G)$, we have $e_H = \varphi(e_G).$
\end{proof}

\begin{proposition}
    A homomorphism maps the inverse element to the inverse of element.
\end{proposition}
\begin{proof}
    Notice that $$\varphi(g)\circledast \varphi(g^{-1})= \varphi(g\odot g^{-1})= e_H \implies \varphi(g)^{-1} = \varphi(g^{-1}).$$
\end{proof}

\begin{definition}[Kernel]
    The kernel of a homomorphism $\varphi: G \to H$ is defined by $$\operatorname{Ker}(\varphi) := \varphi^{-1}(e_H) = \{g\in G: \varphi(g) = e_H\}$$
    \,and it's a normal subgroup of $G$\footnote{To show this, first verify that $\operatorname{Ker}(\varphi)$ is indeed a subgroup. 
    And it's normal as we notice that $\forall k \in g \odot \operatorname{Ker}(\varphi), \exists \mu \in \operatorname{Ker}(\varphi)$ such that $k=g \odot \mu$. 
    Let $\nu=g \odot \mu \odot g^{-1}$, we have $\nu \in \operatorname{Ker}(\varphi)$ and since $\nu \odot g = g \odot \mu = k$, we have $k \in \operatorname{Ker}(\varphi) \odot g.$ The other side can be proved similarly.}.
\end{definition}

\begin{proposition}
    $\varphi$ is injective $\iff \operatorname{Ker} (\varphi)$ is trivial\footnote{That is $\operatorname{Ker}(\varphi)=\{e_G\}$.}.
\end{proposition}
\begin{proof}
    $(\Rightarrow)$ is trivial as injectivity makes $\varphi^{-1}(e_H)$ single-valued and by \propref{prop: identity2identity} we have $e_G \in \operatorname{Ker}(\varphi)$. 

    For $(\Leftarrow)$, suppose there exist $g_1 \neq g_2$ such that $\varphi(g_1) = \varphi(g_2)$. Then $\varphi(e_G) = \varphi(g_2) \circledast \varphi(g_2^{-1})= \varphi(g_1) \circledast \varphi(g_2^{-1})= \varphi(g_1 \odot g_2^{-1}) = e_H$. This contradicts with $\operatorname{Ker}(\varphi) = \{e_G\}$ as $g_1 \odot g_2^{-1}\neq e_G$.
\end{proof}

\subsection{Isomorphisms}
\begin{definition}[Isomorphism]
    An isomorphism is a bijective homomorphism, denoted by $\cong$. If it maps a group to itself, we call it an automorphism.
\end{definition}

\begin{example}[The Set of All Automorphisms as a Group]
    Let $\operatorname{Aut}(G)$ be the set of all automorphisms of $G$, then $(\operatorname{Aut}(G),\circ)$ is a group, which we call it an automorphism group. 
\end{example}

\begin{proposition}[Induced Operation]
    Let $(G,\odot)$ be a group and $H$ is a nonempty set. Given a bijection $\varphi: G \to H$, we define the operation $\circledast$
    $$\mu \circledast \nu := \varphi(\varphi^{-1}(\mu) \odot \varphi^{-1}(\nu)), \mu, \nu \in H$$
    \,then $(H,\circledast)$ is a group and $\varphi$ is an isomorphism.
\end{proposition}
\begin{proof}
    Showing that $(H,\circledast)$ is a group is trivial. To show that $\varphi$ is a homomorphism, given $h_1,h_2 \in H$, there exist $g_1,g_2 \in G$ such that $\varphi(g_1) = h_1$ and $\varphi(g_2) = h_2$. 
    Then by definition, $h_1 \circledast h_2 = \varphi(g_1 \odot g_2)$.
\end{proof}

A very important result to know is that isomorphism hold exactly the same algebraic structure, which provides some convenience in calculation.
For example, when it's hard to deal with $g_1 \odot g_2$, we may turn to $h_1 \circledast h_2$ and use $\varphi^{-1}$ to get $g_1 \odot g_2$.



