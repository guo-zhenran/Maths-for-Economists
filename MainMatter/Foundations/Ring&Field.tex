\section{Ring and Field}
In this section, we add another operations to a group. We require that the set forms an Abelian group with respect to one of the operations and we want a certain kind of distributivity. This leads us to ring.
And we can further add more requirements to rings and make it a field.
\subsection{The Definition}
\begin{definition}[Ring]
    For a 3-tuple $(R,+,\cdot)$, if
    \begin{enumerate}
        \item $(R,+)$ is an Abelian group\footnote{A conventional notation for the identity element is $0_R$.},
        \item $(R,\cdot)$ is a semi-group,
        \item Distributivity: $$a \cdot (b + c) = a \cdot b + a \cdot c,\quad (a + b) \cdot c = a \cdot c + b \cdot c, \qquad \forall a,b,c \in R$$
    \end{enumerate}
    With more specific requirements, we have different kinds of rings:
    \begin{description}
        \item[Commutative Ring] If $\cdot$ is commutative, then $(R,+,\cdot)$ is a commutative ring.
        \item[Ring with Unity] If there exists an identity element $1_R \in R$ for multiplication, then $(R,+,\cdot)$ is a ring with unity. 
    \end{description}
\end{definition}

The definition of subring is similar to that of subgroup.
\begin{definition}[Subring]
    Let $S$ be a nonempty subset of a ring $(R,+,\cdot)$. If $S$ is a subgroup of $(R,+)$ and is closed under multiplication, we say it's a subring.
\end{definition}

\begin{remark}
    Given the existence of additive and multiplicative identity elements in a ring, we know that they're unique. Also, there may exists $a\neq 0_R$ such that $a\cdot b = 0_R$ or $b\cdot a = 0_R$ for some $b\neq 0_R$. Such elements are called \textbf{zero divisors}.
\end{remark}

\subsection{Homomorphism and Isomorphism}
The definition here is similar to that of group homomorphism and isomorphism.
\begin{definition}[Ring Homomorphism and Isomorphism]
    Let $R,S$ be rings. A mapping $\varphi:R\to S$ is a ring homomorphism if it preserve the multiplication and addition operations. If $\varphi$ is bijective, then it is a ring isomorphism.
\end{definition}

Note that if $\varphi$ is a ring homomorphism, then it's a group homomorphism for $(R,+)$ and $(S,+)$ and the kernel is defined by $\operatorname{Ker}(\varphi)=\{r\in R: \varphi(r)=0_S\}$.

\subsection{Ordered Field}
In this subsubsection, we add an order relation to a field and make it an ordered field.
\begin{definition}[Ordered Field]
    An ordered field is a field $(F,+,\cdot)$ equipped with a total order relation $\leq$, if the following statements hold for all $a,b,c \in F$:
    \begin{enumerate}
        \item If $a \leq b$, then $a + c \leq b + c$.
        \item If $0 \leq a$ and $0 \leq b$, then $0 \leq a \cdot b$.
    \end{enumerate}
\end{definition}   