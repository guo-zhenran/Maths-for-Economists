\chapter{Topology Space}
Topology space concerns with the structure of open sets. The first understanding of a open set is from the open ball we defined in the metric space chapter, where we also define the concept of neighborhood.
We'll see that most of the definitions arise from previous concepts, this is how we generalize things.
However, the topology not necessarily relies on metric. We'll see the generality of open sets and return to the topology induced by a metric, where we'll see some interestong results.

\section{the Definition}
\begin{definition}[Topology]
    Given a set $X$ and a family of subsets $\tau$. We say $\tau$ is a topology on $X$ if the followings hold:
    \begin{enumerate}
        \item $\emptyset, X \in \tau$.
        \item $\tau$ is closed under countable union.
        \item $\tau$ is closed under finitie intersection.
    \end{enumerate}
    \,Then we say $\tau$ is a topology and $(X,\tau)$ is a topology space.
\end{definition}

\begin{example}
    We define the following 2 topologies:
    $$\tau_{D}:= \{\emptyset, X\} \qquad \tau_{T} := 2^{X}$$
    \,which is called the discrete topology and the trivial topology.
\end{example}

Notice that $\tau_{D} \subset \tau_{T}$, we say $\tau_{D}$ is (strictly) finer. It's not always the case that we can compare two topology.

Similar to other spaces, we can define its subspace. 
\begin{definition}[Topology Subspace]
    Suppose $Y\subset X$, let $\tau_Y := \{U\cap Y : U \in \tau\}$, then $(Y,\tau_Y)$ is a topology space.
\end{definition}


\subsection{Open Set and Closed Set}
As mentioned before, elements of $\tau$ are called open sets and we define that a closed set has an open complement.

We can now generalize the definition of neighborhood.
\begin{definition}[Neighborhood]
    Suppose $x\in A \subset X$, if there exists an open set $U$ such that $x\in U\subset A$, then we say $A$ is a neighborhood of $x$. If $A$ is an open set itself, we say $A$ is an open neighborhood of $x$.
\end{definition}

Using the language of neighborhood, we can characterize open sets in a diferent way.
\begin{theorem} 
    Suppose $A$ is a subset of $X$, then $A$ is open $\iff \forall x\in A:$ there exists a neighborhood $U_x$ of $x$ such that $x\in U_x$.
\end{theorem}
\begin{proof}
    ($\Rightarrow$) This side is trivial since $A$ is a neighborhood of each $x$ of $X$.

    ($\Leftarrow$) For this side, notice that by definition, there exists an open set $O_x \subset U_x$ for each $x\in X$ such that $x\in O_x$, then 
    $$A=\bigcup_{x\in A} \{x\} \subset \bigcup_{x\in A} O_x \subset \bigcup_{x\in A} U_x \subset A$$
    \,implying $\bigcup_{x\in A} O_x = A$ and it's open.
\end{proof}


We now turn to closed set, by De-Morgan's law, we know that the set $\mathcal{F}$ of all closed sets satisfies:
\begin{enumerate}
    \item $\emptyset,X \in\mathcal{F} $
    \item $\mathcal{F}$ is closed under countable intersection.
    \item $\mathcal{F}$ is closed under finite union.
\end{enumerate}

\subsection{Interior, Closure and Boundary}
Given a set in $X$, it may be open, closed or rather open or closed. This subsection shows how to decompose a certain set to several parts.
\begin{definition}[Interior and Closure]
    Suppose $(X,\tau)$ is a topology space and $A \subset X$, the interior of $A, \Int{A}$, is the biggest open set contained in $A$ and the closure of $A$, $\bar{A}$, 
    is the smallest closed set containing $A$.
\end{definition}

It's equivalent to define $\Int{A}$ as the union of all open subset of $A$ and $\bar{A}$ as the intersection of all closed supersets of $A$. We list the following propositions.
\begin{proposition}
    Suppose $X$ is a topology space and $A,B\subset X$
    \begin{enumerate}
        \item $\Int{A} \subset A \subset \bar{A}$
        \item $x \in A \iff $ there exists a neighborhood $U_x$ of $x$ such that $U_x \in A$.
        \item $A$ is open $\iff A = \Int{A}$; and $B$ is closed $\iff B = \bar{B}$.
        \item $A\subset B \implies \Int{A} \subset \Int{B}, \bar{A} \subset \bar{B}$.
        \item $\Int{(X\setminus A)} = X - \bar{A};\quad \overline{(X\setminus A)} = X - \Int{A}$.
        \item 111 
    \end{enumerate}
\end{proposition}


\subsection{Basis}

\subsection{Topology Induced by Metric}

\section{Continuous Function}


\section{Compactness}
\subsection{the Basics}

\subsection{Compactness of Product Space}

\subsection{Compactness of Induced Topology Space}

\section{Connectedness}










