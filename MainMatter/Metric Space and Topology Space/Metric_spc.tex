\chapter{Metric Space and Convergence}
\section{Metric Space}
The introduction of metric space is to formally state the concept of distance. Consider the distance between 2 points, we naturally require 
it to be non-negative and it equals to 0 if and only if the two points are exactly the same. It should also be symmetric, meaning the distance of $A$ to $B$ is the same as the one of $B$ to $A$.
Moreover, the triangle inequality should also hold, which is a formal illustration of "a straight line is the shortest distance between 2 points". This leads to the definition of metric.
\subsection{the Definitions}
\begin{definition}[Metric Space]
    Let $X$ be a set and we define the metric, a mapping from $X\times X \to \mathbb{R}$, such that 
    \begin{enumerate}
        \item $d(x,y) = 0 \iff x = y.$
        \item Symmetry: $\forall x,y \in X : d(x,y) = d(y,x).$
        \item Triangle inequality: $\forall x,y,z \in X :d(x,y) \leq d(x,z)+d(z,y).$
    \end{enumerate}
\end{definition}
\begin{remark}
    The third axiom can be extended as $$\forall x,y,z \in X :d(x,y) \geq |d(x,z)-d(z,y)|.$$
    This holds by noticing that $d(x,y) \geq d(x,z) - d(y,z)$ and interchanging $x$ and $y$, then
    $$d(x,y) = d(y,x) \geq d(y,z) - d(x,z) = - [d(x,z) - d(y,z)]$$
    \,This leads to the result.
\end{remark}

We show some special metrics.
\begin{example}[the Discrete and Natural Metric]
    We call the following metrics discrete metric and natural metric respectively,
    $$d_D : (x,y) \in X\times X \mapsto \delta_{x,y} \qquad d_N : (x,y) \in X\times X \mapsto |x-y|$$
    \,where $\delta_{x,y}$ means
    $$\delta_{x,y} = \begin{cases}
        0, & x=y \\
        1, & x\neq y
    \end{cases}$$
\end{example}

\subsubsection{Diameter}
The previous defition is concerning 2 points, we now extend it to the sets.
\begin{definition}[Diameter]
    For a subset $Y$ of $X$,
    $$\diam(Y) := \sup_{x,y \in Y} d(x,y).$$
    \,If it's finite, we say $Y$ is $d$-bounded or simply bounded when the metric is clear form the context.
\end{definition}


\subsubsection{Neighborhood}
We introduce the concept of neighborhood for a metric space, before which we introduce ball. We'll see the general definition of neighborhood when encontering topology.
\begin{definition}[Ball]
    An open ball with center at $a$ and radius $r$ is defined as $$\oball(a,r):= \{x\in X : d(a,x) < r\}$$
    \,and a closed ball with center at $a$ and radius $r$ is defined as $$\cball(a,r):= \{x\in X : d(a,x) \leq r\}$$
\end{definition}

\begin{definition}[Neighborhood]
    We say $U\subset X$ is a neighborhood of $a\in X$ if $\exists r \in \R : \oball(a,r)\subset U.$ And we denote $\mathcal{U}(a)$ as the set of all neighborhoods of $a$.
\end{definition}

\subsubsection{Cluster Point}
We can define cluster point for a sequence.
\begin{definition}[Cluster Point of a Sequence]
    Suppose $a\in X$, we call it a cluster point of $\{x_n\}_{n=1}^{\infty}$ if $\forall U \in \mathcal{U}(a): U$ contains infinitely many terms of $\{x_n\}_{n=1}^{\infty}$.
\end{definition}

The following proposition is straightforward by definition.
\begin{proposition}
    The followings are equivalent:
    \begin{enumerate}
        \item $a$ is a cluster point of $\{x_n\}_{n=1}^{\infty}$.
        \item $\forall U\in \mathcal{U}(a)$ and $m\in \N: \exists n>m$ such that $x_n\in U$.
        \item $\forall \varepsilon \in \R$ and $m\in \N: \exists n>m$ such that $x_n\in \oball(a,\varepsilon)$.
    \end{enumerate}
\end{proposition}

This can be extended to a set as well.
\begin{definition}[Cluster Point of a Set]
    Let $E \subset X$ and we say $x_0$ is a cluster point of $E$ if $\forall \delta \in \R^+ : \exists x\in E\setminus \{x_0\}$ such that $d(x_0,x)< \delta$. 
    Or equivalently, $\forall \delta \in \R^+ : \exists r\in \R$ such that $(E\setminus \{x_0\}) \cap \oball(x_0,r) \neq \emptyset.$
\end{definition}


\subsection{Induced Metric}


\section{Convergence}
\subsection{Convergence of a Sequence}
What is convergence? An intuitive explaination is that when $n$ goes large enough, the sequence gets closer and closer to a certain number $a$.

\begin{definition}[Convergence of a Sequence]
    For a sequence $\{x_n\}_{n=1}^{\infty} \subset X$, suppose there exists $x\in X$ such that 
    $$\forall \varepsilon>0 : \exists N= N(\varepsilon)\in \N \text{ such that } d(x_n, x) <\varepsilon,\forall n >N.\footnote{Or equivalently, $x_n \in \oball(x,\varepsilon),\forall n >N.$}$$
    In such case we say $x$ is the limit of $\{x_n\}$, denoted by $x_n\to x(n\to \infty)$ or $\lim_{n\to \infty} x_n = x.$
\end{definition}

The following proposition is trivial by definition.
\begin{proposition}
    Suppose $\{x_n\} \subset X$, the followings are equivalent:
    \begin{enumerate}
        \item $x_n \to x (n\to \infty)$
        \item Each neighborhood of $x$ contains almost all terms of $\{x_n\}$, that is the complement of each neighborhood contains at most finite terms of $\{x_n\}$.
        \item $\forall U \in \mathcal{U}(x) : \exists N = N(U)$ such that $x_n \in U, \forall n > N$.
    \end{enumerate}
\end{proposition}

\begin{remark}
    It's clear that a limit is a cluster point, but a cluster point is not necessarily a limit. It's sometimes easier 
    to show a sequence doesn't converge to $a$ by showing $a$ is not a cluster point.
\end{remark}


The following 2 propositions are proved using the triangle inequality.
\begin{proposition}[Every Convergent Sequence is Bounded]
    Suppose $x_n \to x (n\to \infty)$, then $\{x_n\}$ is bounded.
\end{proposition}
\begin{proof}
    By definition, there exists $N\in\N$ such that $\forall n >N : d(x_n,x)<1$. Thus,
    $$d(x_n,x_m) < d(x_n,x) +d(x_m,x) <2,\quad \forall n,m \geq N$$
    \,Let $M:=\max_{i,j \leq N}\{2,d(x_i,x_j)\}$, then $2M$ is a bound by noticing that 
    $d(x_n,x_m) < d(x_n,x_N) +d(x_m,x_N)$.
\end{proof}

\begin{proposition}[Uniqueness of Limit]
    Suppose $x_n \to x (n\to \infty)$, then $x$ is the unique cluster point of $\{x_n\}$.
\end{proposition}
\begin{proof}
    Suppose there exists $x^{\prime}$ such that $x_n \to x (n\to \infty)$. We try to construct a ball which contains finite terms of $\{x_n\}$.

    Let $\epsilon := d(x,x^{\prime})/2$. Since $x_n \to x (n\to \infty)$, we have $\exists N = N(\epsilon): d(x,x_n) < \epsilon,\forall n>N$ dy definition. Then 
    $$\forall n > N : d(x_n,x^{\prime}) \geq |d(x,x_n) - d(x,x^{\prime})| > \epsilon$$ 
    \,meaning there exists at most finitely many terms of $\{x_n\}$ in $\oball(x^{\prime}, \epsilon)$.
\end{proof}

\begin{corollary}
    Every convergent sequence has a unique limit.
\end{corollary}
With this corollary, we show our definition is well-defined.

\section{Improper Convergence}
From now on we narrow it down and consider sequences defined on real lines.

We calculate an example to show how to categorize sequence in terms of convergence and divergence.
\begin{example}
    Consider the following 3 sequences.
    $$a_n = \frac{1}{n},\quad b_n = n, \quad c_n = \begin{cases}
        a_n,& n = 2m\\
        a_n + 1,& n \neq 2m
    \end{cases}$$
\end{example}
It's trivial that $a_n$ converge to $0$ and $a_n+1$ converge to $1$. This leads to $c_n$ being divergent since a convergent sequence should have a unique limit.
However, we can't find a suitable limit for $b_n$.
\subsection{Infinite Limit}
In this subsection, we're to extend the metric on $\R$ so that sequences like $b_n$ can be given a limit.
\begin{definition}[Extended Concepts Concerning $\pm \infty$]
    Let $\bar{\R} := \R \cup \{\pm \infty\}$, the extended concepts are defined as:
    \begin{description}
        \item[Neighborhood] a subset $U$ is called a neighborhood of $+ \infty$ if there exists $K\in \R^+$ such that $(K,+\infty) \subset U$ (or $(-\infty,-K)$ for $-\infty$). 
        \item[Cluster Point] $\pm \infty$ is called the cluster point of $\{x_n\}$ if each neighborhood of $\pm \infty$ contains infinitely many terms of $\{x_n\}$.
        \item[Convergence to $\pm \infty$(Improper Convergence)] $x_n\to \pm \infty (n\to \infty)$ is defined as: $$\forall \varepsilon>0 : \exists K\in \R^+ \text{ such that } |x_n|\geq \varepsilon, \forall n >K.$$
    \end{description} 
\end{definition}

\subsection{Some Special Sequences}
\subsubsection{Monotone Sequence}
\begin{theorem}[Monotone and Bounded]
    
\end{theorem}

\begin{theorem}[Stolze]
    
\end{theorem}

\subsubsection{Subsequence}
\begin{theorem}[Bolzano-Weierstrass]
    
\end{theorem}

\subsection{Limit Superior and Limit Inferior}
This subsection is to describe sequences like $b_n$, with which we can give it a limit.

\begin{definition}[Limit Superior and Limit Inferior]
    We construct the following 2 sequences:
    $$
        \bar{x}_n := \sup_{k\geq n} x_n \qquad \underline{x}_{n} := \inf_{k\geq n} x_n
    $$
    \,We know by definition that they are monotonic, thus converging in $\bar{\R}$, whose limit is defined by 
    $$\limsup_{n\to \infty} x_n := \lim_{n \to \infty} \bar{x}_n \qquad \liminf_{n\to \infty} x_n := \lim_{n \to \infty} \underline{x}_{n}.$$
\end{definition}

The next theorem shows how limit inferior and superior dipict cluster points.
\begin{theorem}[Limit Inferior(Superior) is the Smallest(Greatest) Cluster Point]
    Let $x_*$ and $x^*$ be the smallest and greatest cluster point. Then
    $$\limsup_{n\to \infty} x_n = x^* \qquad \liminf_{n\to \infty} x_n = x_*.$$
\end{theorem}
\begin{proof}
    
\end{proof}

Suppose $\{x_n\}$ converges in $\R$, it must have a unique cluster point, meaning $\limsup_{n\to \infty} x_n = \liminf_{n\to \infty} x_n$. 
We'll show in the next theorem that the converse statement also holds.
\begin{theorem}
    Suppose $\{x_n\}$ is a sequence in $\R$, then 
    $$x_n\to x (n\to \infty) \iff \limsup_{n\to \infty} x_n = \liminf_{n\to \infty} x_n = x.$$
\end{theorem}
\begin{proof}
    
\end{proof}

\section{Completeness}
\subsection{Cauchy Sequence}
\subsection{Banach Space}