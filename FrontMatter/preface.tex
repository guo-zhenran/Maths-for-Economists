\chapter*{Preface}

This book is written as a handbook for students of economics rather than a course note for a specific subject like MathEcon. 
To make it a comprehensive one, I'll try to cover almost all maths needed for studying economics, including logic, algebra, analysis, topology, probability theory and optimization theory. 
Furthermore, I will incorporate additional topics, drawing primarily from the syllabi of courses I have completed at SUFE.
In this way, those (especially from SUFE) who are taking courses like Mathematical Analysis can refer to it as a supplementary resource.
However, considering the vast range of knowledge required, it's NOT possible to cover everything, thus the book is written with an emphasis on the mathematical foundations for economics.

Though serving as a handbook, the book will be written as intuitively as possible. I believe this book is suitable for self-taught learners as we're biulding all the knowledge step by step while reading it. 
This approach, I believe, mirrors the way many economists think: by providing an intuitive understanding of how different branches of knowledge are developed and structured as a whole, the definitions and techniques can feel natural and logical to the reader.

I recommend treating this book, which is written in a modern approach, as a second course after you complete your first year of study. A solid grasp of the basics of linear algebra and mathematical analysis is required as prerequisites so that the modern approach is understandable. The book is under construction and may be updated on a monthly basis.

